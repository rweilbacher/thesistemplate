% Chapter 1

\chapter{Detailed guidance on the hypervisor architecture} % Main chapter title

\label{Chapter5} % For referencing the chapter elsewhere, use \ref{Chapter1} 

%----------------------------------------------------------------------------------------

% Define some commands to keep the formatting separated from the content 
\newcommand{\keyword}[1]{\textbf{#1}}
\newcommand{\tabhead}[1]{\textbf{#1}}
\newcommand{\code}[1]{\texttt{#1}}
\newcommand{\file}[1]{\texttt{\bfseries#1}}
\newcommand{\option}[1]{\texttt{\itshape#1}}

%----------------------------------------------------------------------------------------
After the free form analysis of chapter \ref{Chapter4} it is now time to give more concrete examples on when and why the hypervisor architecture should be chosen. For this purpose several \keyword{reference projects} will be laid out and it will be discussed what makes them good or bad examples for the architecture. Afterwards some basic guidance will be given on how a specific third-party hypervisor implementation should be chosen and what metrics need to be gathered.
\section{Reference projects}
* Imaginary!!!
* Requirements and goals are chosen to align with the message. There may be parts that are unrealistic but a manufacturer consulting this thesis for advice should be able to draw parallels to his upcoming and ongoing projects. 
*  As in \ref{Chapter3} only the hypervisor and hardware separated subsystems architectures will be considered. 
* Obviously simplified
* Mention that metrics like cost etc. can't really be considered
* Intended as some examples not a definitive decision making process.
* Trivial cases are not examined because they don't warrant the effort of explaining an entire device architecture.
* Say what the preconditions for a reference project are
\subsection{Portable gas detector}
% https://www.elektroniknet.de/markt-technik/elektronikfertigung/perfekte-symbiose-von-oem-und-ems-151254.html
\subsubsection{Introduction}
This example project is a general, portable gas detector that measures the concentration of different gases by detecting the color change of a chemical, sensitive to the gas in question. Cartridges with clear, sealed tubes can be inserted into the device and once a measurement begins, one of the tubes is punctured and air is pumped through at a specific rate. Then the color change is observed by a \gls{CMOS} sensor. Because the color change, pump speed and other metrics are different for every gas, information can be extracted from the cartridge via an \gls{NFC} tag. This tag also contains general information about the cartridge in question.
% TODO Mention device use environment

In this thought experiment the first version of this device has already been on the market for some time and now the manufacturer is looking to create the next generation. The current generation is equipped with an ARM Cortex-M4, a USB port for transmitting measurement data and an LCD display with a simplistic \gls{GUI}. All of the software sits on top of a small \gls{RTOS}, with the \gls{GUI} framework also being provided by the \gls{RTOS} developer.

In the upcoming generation the manufacturer wants to expand the device's functionality by adding Bluetooth support and a more flexible \gls{GUI} . Furthermore, the manufacturer wants to be able to extend both the safety-critical and non-safety-critical functionality without having to release a new hardware revision. The core gas measurement code is hardware independent and already verified. Therefore, it could be included in the next generation with minimal effort, if the code's integrity can be guaranteed.
% NOTE How important is unit cost?
* Mid criticality project
* Fairly low complexity but slick design and nice GUI are required.
\subsubsection{Discussion}
% TODO I only mention SWaP but SWaP-C is probably improved. At least in terms of direct hardware cost!
The current device generation uses no separation architecture because it is not very complex and any components of lower criticality were feasible to achieve within the chosen architecture. With the upcoming generation there are now stronger initial incentives to consider the separation of software components:
\begin{enumerate}
\item There is already verified code available that the manufacturer wants to reuse with little to no modification.
\item \gls{GUI} libraries on Linux would be a great fit for realizing a more complex \gls{GUI} with less effort than on a typical \gls{RTOS}. But Linux can not be feasibly verified for the safety-critical use case.
\item Both safety-critical and non-safety-critical functionality will likely be extended in the future. As established in section  \ref{safety-analysis}, granular partitions or the ability to add more can make regression testing cheaper and less error-prone.
\end{enumerate}
Additionally, the planned features do not directly impact the operators safety. The only component that can pose a threat is the \gls{GUI} because the display could show an incorrect concentration. To mitigate this a red \gls{LED}, controlled by the critical code, will begin blinking if the measured concentration is above the limit specified on the cartridge's \gls{NFC} tag. Any other failure in these components will either just prohibit retrieving data from the device for later analysis or render the device unusable. And while these effects are undesirable, the operator can recognize them and take actions to protect himself.

% NOTE References to relevant sections?
So now that the manufacturer has come to the conclusion that a separation architecture would be beneficial for the device, he needs to find one that supports his use case. For a battery powered, hand-held device \gls{SWaP} are all very important and this provides a big advantage for the hypervisor, as the \gls{HSS} architecture compares unfavourably in regards to these criteria. Of course using the Linux \gls{GUI} offerings would invariably lead to a stronger processor being necessary to support Linux but this would be the case for both separation architectures and for this case we will assume the manufacturer has decided that this is a worthwhile drawback for the increased graphical fidelity. 

Separating the gas measurement code from other parts of the software, at least the non verified parts, is equally doable on both architectures. However, the hypervisors support for more granular partitions makes regression testing of a software revision cheaper and can generally aid the software extendibility and maintainability.

One last benefit the hypervisor has in this scenario is established by its hardware abstraction. Any software components that are created for the next device generation will be more likely to be reusable in any future generations, with only slight or no modifications.
% TODO Maybe mention advanced safety features?
\subsubsection{Conclusion}
This project exemplifies a case where separation is becoming increasingly attractive because of growing consumer demands and processing power but where the \gls{HSS} architecture violates the core demands of the use case. This category is likely to grow is very likely to grow in the future[source].

With the criticality level of the device and the specific risks involved, the hypervisor architecture's lack of support for advanced safety features and its less provable separation are no death sentence. And finally, the hypervisors malleability lends itself well to creating a future-proof design, if unit cost is not a bigger concern. 
% TODO Is there more I want to say?
\subsection{Automated blood diagnostics device}
\subsubsection{Introduction}
% TODO Describe informally what the machinery does and what it consists of
This reference project is an upcoming in Vitro diagnostics device that analyzes a patients blood. Its functionalities include automatic test processing, result interpretation and data management. Blood assays include both standard blood characteristics, as well as disease screening.
The device's use environment are clinical blood labs and donation centers. It allows an operator to define and schedule automatic assays on top of displaying results on the attached operator station. The operator station consists of a monitor, a computer-mouse and a keyboard. Furthermore, the device is connected to a database of blood data for the synchronization of results.

There is a clear divide of criticality between the operator station and the components responsible for handling and analyzing the blood samples. For the operator station it is important that the assay configuration gets sent to analysis component correctly and that the test results received after an analysis are correct. Transmitting and receiving results from the database is also not critical to the patients safety.

% TODO Mention in Vitro medical standard 
The blood analysis component on the other hand is more critical. A perfect uptime is not required as a patient is not immediately dependent on the device. Instead, it is much more important that faults or incorrect measurements are discovered and communicated to an operator. Cross-contamination also needs to be avoided at all costs.

\subsubsection{Discussion}
With this device, separation is almost mandatory because the operator station contains a large low-criticality \gls{GUI} that can be most optimally realized on a general purpose operating system, while the critical part needs to reliably control motors and react to input within a short time-frame. Traditionally this system would have been realized using the \gls{HSS} architecture so now the question remains whether or not a hypervisor can be used effectively instead.

First of all, due to the nature of the device, perfect uptime is not required, this means that redundancy is also not essential. This works in the favor of the hypervisor, since in the \gls{HSS} architecture only the safety-critical component could be made redundant but that option would not be available for a hypervisor because all partitions are part of one processor. 

However, the system's need for fault tolerance means that some advanced fault tolerance techniques are necessary to mitigate the risks as much as possible. One way this could be achieved with a hypervisor architecture can be seen in figure [ref]. In this case both the safety-critical partition and the operator station partition are running on the hypervisor, with a secondary microcontroller performing the same calculations and verifying the output of the safety-critical hypervisor partition. Even though this means there are now multiple processors present, the hypervisor still provides consolidating effects. A pure \gls{HSS} architecture would still require a minimum of three partitions for the same setup.
% FIGURE Make an architecture figure 

% NOTE The proximity things seems like bullshit. Where did I even get that from?
If we assume that the separation is adequate and all basic safety requirements can be met, there are still factors left to consider. First of all, since this is a system that deals with real-time requirements, proximity to the machinery could be valuable. But in this case \gls{EMC} can cause significant issues. In the hypervisor setup a processor with a higher frequency would sit closer to the motors and other machinery which also themselves have \gls{EMI}. This could easily lead to a violation of the \gls{EMC} requirements of the regulatory body, introducing the need to add more electromagnetic shielding. 

This device can also benefit from the hypervisor's architectural improvements and easier configuration, perhaps even more so. Since it is more complex, it is also more likely that functionality can be usefully extracted into partitions, carrying the benefits explained in \ref{Chapter4}. But it is important to keep in mind that for a device with these characteristics, some of the typical hypervisor benefits are less relevant.

First of all, the hardware cost savings are not going to make a big impact on a device this complex and big. Secondly, \gls{SWaP} are also not as crucial as for the portable gas detector. The housing of this unit is already going to be at least as big as a typical wardrobe and an extra microcontroller is not going to change that. Same goes for weight. And since the device is connected to power all the time, making it as power efficient as possible is also not a primary goal.

\subsubsection{Conclusion}
This example represents a high complexity, high criticality project that can already profit from the more traditional approaches of separation but where, in a reevaluation of device architecture, the hypervisor approach might come out on top. 

* But consequently it is also a more competitive choice than, for example the portable gas detector example.
* SWaP-C is not hugely important but that does not mean lower SWaP-C is not desirable. (Not really SWaP-C, see discussion)
* Along with the other hypervisor benefits [list them in discussion] it can still be an overall better choice, granted the secondary factors like EMC, proximity and safety of separation can all be provided, as well as the hardware platform that can support a hypervisor being suitable for the hardware tasks that need to be performed.
* As might have come apparent in the discussion section for this project, with growing device complexity, the complexity of the decision grows rapidly as well. There are factors beyond the safety of separation that can make the hypervisor architecture muss less effective. Ultimately, which factors are important and how much needs to be decided on the basis of an actual project and this examination is nothing but an exploration how things could be decided.

%----------------------------------------------------------------------------------------

\section{How to decide on, a hypervisor implementation}
* You are already seriously considering using a hypervisor or maybe you already know that you want to use one but are not sure which implementation. This section with will summarize all of the relevant aspects and what to look for. 
* This analysis may reveal that there is no third-party hypervisor that can satisfy your cost targets and the architecture may not be right for that project
* This section applies to choosing a third-party hypervisor as a first-party one can provide whatever you want it to and are willing to invest into it.
* This section will be mostly a list of questions to ask with further explanation if deemed necessary    
\subsection{Certifiability}
* What does the hypervisor developer promise?
* What deliverables does he offer to aid with this?
* What services does he offer?
* Has the hypervisor been used in a device that has to comply to the same regulations?
* Do the claims of the developer and his documentation survive more intense scrutiny and project specific risk analysis?
\subsection{Cost}
* What license model is applicable?
* Is there functionality or services that cost extra? Are they desirable?
* Getting a quote from the developer for the project.
* Negotiation
* What costs are associated with the OS that will run on the hypervisor?
\subsection{Dependence}
* Is the source code available?
* Is it possible to enlist the developer to add more functionality or change existing?
* Is it possible to modify the hypervisor alone?
* Is it feasible to create BSP without the developer?
* If paravirt is used, what OSes are supported and can a custom OS feasibly be adjusted for the hypervisor?
* 
\subsection{Basic functionality}
* What basic functionality can the developer offer?
Examples are:
	* SMP and AMP
    * Partition communication (And how that is achieved)
    * Possible scheduling policies
    * Device pass-trough (Probably requires additional virtualization hardware)
    * Virtual device support
    * Possible guest environments
    * Para- and full virtualization
    * Hardware support
    
\subsection{Additional functionality}
* Is the hypervisor bare-bones and only offers basic separation or does it have more features?
Interesting features are:
	* Health monitoring and automatic action (for example different scheduling policy
    * Security 
    * Automatic testing (Lynx BIT)
    * Virtual high resolution timers (Kind of solves an issue the hypervisor created)
    * 
\subsection{Impact on the development process}
\subsubsection{Tooling}
* What tools are available for configuration?
* What tools are available for the creation of images?
* What compilers and debuggers are supported and do they align with the project requirements?
* CLI or IDE?
* How is configuration performed and is it appropriately easy?
\subsubsection{Debugging and tracing}
* Can the hypervisor be debugged at all?
* Can running partitions be debugged?
* Compared to a typical debugging process, does the hypervisor help or hinder debugging?
* Does the hypervisor allow for tracing of hypervisor system events?
* 

%----------------------------------------------------------------------------------------

\section{Noteworthy implementations}
* (LynxSecure)
* PikeOS (integrated RTOS)
* seL4 (formally verified but very academic)
* Mentor (as a really small example, maybe find additional)
* OpenSynergy micro-hypervisor
* PROVENVISOR (formally verified, security focus)
* GreenHills, WindRiver and maybe QNX as "expensive high-profile" hypervisors
* [...]