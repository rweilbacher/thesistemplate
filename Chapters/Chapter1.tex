% Chapter 1

\chapter{Introduction} % Main chapter title

\label{Chapter1} % For referencing the chapter elsewhere, use \ref{Chapter1} 

%----------------------------------------------------------------------------------------

% Define some commands to keep the formatting separated from the content 
\newcommand{\keyword}[1]{\textbf{#1}}
\newcommand{\tabhead}[1]{\textbf{#1}}
\newcommand{\code}[1]{\texttt{#1}}
\newcommand{\file}[1]{\texttt{\bfseries#1}}
\newcommand{\option}[1]{\texttt{\itshape#1}}

\section{Goals}

%----------------------------------------------------------------------------------------

\section{Methods}

%----------------------------------------------------------------------------------------

\section{Safety-critical devices}
\subsection{Basic concepts}
\subsubsection{Safety}
\subsubsection{Regulations}
\subsubsection{Criticality levels}
\subsection{Challenges}
\subsubsection{Mixed criticality}

%----------------------------------------------------------------------------------------

\section{The separation architecture}
\subsection{Spatial separation}
\subsection{Temporal separation}
\subsection{Independence}
\subsection{Certifiable compliance}

%----------------------------------------------------------------------------------------

\section{Existing architectures}
\subsection{Embedded operating systems}
\subsubsection{GPOS}
\subsubsection{RTOS}
\subsection{AMP}
\subsection{Hardware supervised AMP}
\subsection{Hardware separated subsystems}

%----------------------------------------------------------------------------------------

\section{Embedded safety hypervisor origins}
\subsection{Virtualization}
\subsection{Original hypervisor use case}
\subsection{Microkernels}
\subsection{Unification of the two concepts}


